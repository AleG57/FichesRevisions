\chapter*{Introduction}
Les objectifs du cours sont de présenter les bases de la théorie de la décision :
\begin{itemize}
    \item \textbf{L'estimation pure}
    \item \textbf{La détection} \newline
\end{itemize}

Plan du cours : \newline
\begin{itemize}
    \item Notions d'estimation
    \begin{itemize}
        \item Introduction
        \item Estimateurs bayésiens (espaces de Hilbert, projection orthogonale, estimation moyenne quadratique  avec contrainte linéaire)
        \item Estimateurs non bayésiens (Inégalité de Cramer-Rao, maximum de vraisemblance)
    \end{itemize}
    \item Estimation d'un signal dans un bruit auditif
    \item Analyse spectrale non paramétrique
    \item Détection
    \begin{itemize}
        \item Test des hypothèses (théorie bayésienne, stratégie de Nayman-Person, courbes COR)
        \item Application à la détection du signal dans un bruit (décomposition de Karhunen-Loève, détection d"un signal déterminisme dans un bruit gaussien)
    \end{itemize}
    \item Détection
    \item Filtrage linéaire statistique
    \begin{itemize}
        \item Introduction
        \item Filtrage de Wiener 
        \item Filtrage de Wiener avec filtrage linéaire
    \end{itemize}
    \item Prédiction à un pas et passé infini
    \begin{itemize}
        \item Cas d'un signal dont la densité spectrale de puissance est bornée et admet une factorisation forte
        \item Cas général, décomposition de Wold
    \end{itemize}
    \item Interpolation d'un signal stationnaire
    \item Prédiction à un pas passé et infini
    \item Primitives de la théorie de l'information
    \begin{itemize}
        \item Introduction (source d'information discrète, canal discret, message)
        \item Deux problèmes clés de codage (codage canal, codage source distribuée)
        \item Théorèmes fondamentaux (codage aléatoire (random coding), compartimentage aléatoire (random binning))
    \end{itemize}
    \item Exercices de la théorie de l'information (2H de TD) \newline
\end{itemize}
\newpage
\noindent Prérequis : \newline
\begin{itemize}
    \item Cours de CIP
    \item Cours de Traitement du Signal
    \item Cours de Statistique et Apprentissage
    \item COurs de SIP
\end{itemize}