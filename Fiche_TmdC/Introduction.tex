\chapter*{Introduction}
Les objectifs du cours sont de présenter les bases de la théorie de la décision :
\begin{itemize}
    \item \textbf{L'estimation pure}
    \item \textbf{La détection} \newline
\end{itemize}

Plan du cours :
\begin{itemize}
    \item Notions d'estimation
    \begin{itemize}
        \item Introduction
        \item Estimateurs bayésiens (espaces de Hilbert, projection orthogonale, estimation moyenne quadratique  avec contrainte linéaire)
        \item Estimateurs non bayésiens (Inégalité de Cramer-Rao, maximum de vraisemblance)
    \end{itemize}
    \item Estimation d'un signal dans un bruit auditif
    \item Analyse spectrale non paramétrique
    \item Détection
    \begin{itemize}
        \item Test des hypothèses (théorie bayésienne, stratégie de Nayman-Person, courbes COR)
        \item Application à la détection du signal dans un bruit (décomposition de Karhunen-Loève, détection d"un signal déterminisme dans un bruit gaussien)
    \end{itemize}
    \item Détection
    \item Filtrage linéaire statistique
    \begin{itemize}
        \item Introduction
        \item Filtrage de Wiener 
    \end{itemize}
    \item Filtrage de Wiener avec filtrage linéaire
\end{itemize}
Prérequis :
\begin{itemize}
    \item 
\end{itemize}