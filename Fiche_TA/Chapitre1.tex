\chapter{Les sons - Modèle de perception}
\section{Introduction}
\subsection{Définitions}
\begin{itemize}
    \item \textbf{Psychophysique} : relation entre le \textit{stimulus}\footnote{Phénomène physique} et la \textit{sensasion} ressentie du stimulus. \newline
    \item \textbf{Psychoacoustique} \footnote{Remarque : on peut tromper l'ouie comme la vue, avec des sons appelés \textbf{sons de Risset}} : étude de la relation entre les vibrations des ondes sonores et sa perception.\newline
    \item \textbf{Les modèles de production} permettent de caractériser les osurces dans la nature.
\end{itemize}
\subsection{Histoire des sens} \footnote{D'après Ch. Sherrington (1857-1952)}
\begin{itemize}
    \item Les sens \textit{introseptifs} : sensations qui vienne des entrailles du corps (estomac, coeur, malaise, aise...)
    \item Les sens \textit{Proprioceptifs} : 
        \begin{itemize}
            \item Sens \textit{statique} ou \textit{labyrinthique}\footnote{Provient du "capteur" situé dans l'oreille interne} : mouvements de rotation et de translation
            \item Sens \textit{kinésique} ou \textit{kinestésique} : permet la perception des objets dans l'espace, par exemple le toucher
        \end{itemize}
    \item Les sens \textit{extéroceptifs}
        \begin{itemize}
            \item Sens par contact direct
                \begin{itemize}
                    \item Le \textit{toucher}
                    \item Les sens \textit{chimique} : goût, odorat
                \end{itemize}
            \item Sens par contact indirect
                \begin{itemize}
                    \item \textit{Vue}
                    \item \textit{Ouie}
                \end{itemize}
        \end{itemize}
\end{itemize}
\newpage
\subsection{Lois des sens (18ème - 19ème)}
\begin{itemize}
    \item \textbf{Loi du sens} : il existe pour chaque sens une intensité minima du stimulus, appelée intensité liminaire, au-dessous de laquelle il n'y a pas de sensation
    \item \textbf{Loi du seuil différentiel} : 
        \begin{itemize}
            \item Forme a. \newline
            Il existe un rapport constant entre l'intensité du stimulus initial et la variation minima qu'il faut lui faire subir pour que la différence soit sentie
            \item Forme b. \newline
            Pour que la sensation subisse des accroissements en progression arithmétique (0, 1, 2...), il faut faire varier le stimulus en progression géométrique (a, a2, a3...) ; le rapport constant est le seuil liminaire. C'est encore la loi logarithmique, ou loi de Fechner
        \end{itemize}
\end{itemize}
\newpage
\section{Stimulus auditif : le son}
\subsection{Qu'est-ce que le son ?}
C'est la sensation perçue par l'oreille. Variation périodique de la pression d'un milieu.
\subsection{Hypothèses du cours}
\begin{itemize}
    \item Millieux de propagation \footnote{(gazs, liquides, solides)} supposés parfaits, sans viscosité et au repos \footnote{En réalite, pour les fluides visqueux, on doit résoudre l'équation de Navier-Stokes par la méthode des éléments finis}
    \item Vibrations de faible amplitude
    \item Transformations des fluides supposés adiabatiques réversibles
\end{itemize}
\subsection{Equation d'ondes unidimensionnelle}
Equation de propagation des ondes électro-magnétiques :
\begin{equation}
    \large
    \tcbhighmath[fuzzy halo=0.5mm with electricultramarine!35!electricultramarine,arc=2pt,
    boxrule=0pt,frame hidden]{ 
        \frac{\partial^{2} \psi}{\partial t^{2}} = c_{s}^{2}\frac{\partial^{2} \psi}{\partial x^{2}}
     } \nonumber
    \normalsize
\end{equation}
et : 
\begin{equation}
    \label{eq 1}
    \rho_{e} = -\rho_{0}\frac{\partial \psi }{\partial x}  
\end{equation}
\begin{equation}
    \label{eq 2}
    P_{e} = c_{s}^{2}\rho_{e}
\end{equation}
\begin{equation}
    \label{eq 3}
    \rho_{0} \frac{\partial^{2} \psi}{\partial t^{2}} = -\frac{\partial P_{e}}{\partial x}
\end{equation}
Ce qui devient pour l'équation d'ondes sonores :
\begin{equation}
    \Large
    \tcbhighmath[fuzzy halo=0.5mm with electricultramarine!35!electricultramarine,arc=2pt,
    boxrule=0pt,frame hidden]{ 
        \frac{\partial^{2} P_{e}}{\partial t^{2}} = \frac{1}{c_{s}^{2}}\frac{\partial^{2} P_{e}}{\partial x^{2}}
     } \nonumber
    \normalsize
\end{equation}
\vfil
\begin{equation}
    \large
    \tcbhighmath[fuzzy halo=0.5mm with electricultramarine!35!electricultramarine,arc=2pt,
    boxrule=0pt,frame hidden]{ 
        c_{s} = 331,4\sqrt{1+\frac{T}{T_{0}}}
     } \nonumber
    \normalsize
\end{equation}
\vfill
\begin{center}
    \small{Pour démonstration voir annexe \ref{Annexe 1}}
\end{center}
\newpage
\subsection{Equation d'ondes en tridimensionnel}
L'équation d'ondes s'écrit à l'aide du d'Alembertien : 
\begin{definition}{D'Alembertien}{d'Alembertien}
    \[ \Xi = \nabla^{2} - \frac{1}{c_{2}^{2}}\frac{\partial^{2}}{\partial t^{2}} \] Avec  : \[ \Xi P_{e} = 0 \]
\end{definition}
Soit :
\begin{equation}
    \large
    \tcbhighmath[fuzzy halo=0.5mm with electricultramarine!35!electricultramarine,arc=2pt,
    boxrule=0pt,frame hidden]{ 
        \nabla^{2}P_{e} - \frac{1}{c_{2}^{2}}\frac{\partial^{2} P_{e}}{\partial t^{2}} = 0 \nonumber
     } \nonumber
    \normalsize
\end{equation}
Avec : \newline
\large
\begin{itemize}
    \item $\nabla^{2}\psi = \Delta \psi = \frac{\partial^{2} \psi}{\partial x^{2}} + \frac{\partial^{2} \psi}{\partial y^{2}} + \frac{\partial^{2} \psi}{\partial z^{2}} + \frac{\partial^{2} \psi}{\partial t^{2}}$ \newline
    \item $\psi$ : fonction d'onde (électromagnétique) \newline
    \item $c_{s}$ : célérité du son dans l'air en $m.s^{-1}$ ($c_{s} \approx 340 m.s^{-1}$) \newline
    \item $P_{e}$ : pression acoustique en $Pa$ \newline
    \item $T$ : température en $C$ \newline
    \item $T_{0} = 273 C$
\end{itemize}
\normalsize
\vfil
Quelques rappels : 

\begin{equation}
    \normalsize
    \tcbhighmath[fuzzy halo=0.5mm with electricultramarine!35!electricultramarine,arc=2pt,
    boxrule=0pt,frame hidden]{ 
        \lambda = cT = \frac{c}{f}
     } \nonumber
    \normalsize
\end{equation}
\begin{itemize}
    \item $c$ : célérité de la lumière ($c = 3.10^{8} m.s^{-1}$)
    \item $T$ : période en $s$
    \item $f$ : fréquence en $Hz$
\end{itemize}
\newpage
