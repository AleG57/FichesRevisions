\chapter*{Annexe 1 : démonstration de l'équation d'ondes}
\label{Annexe 1}
On considère \eqref{eq 2} \newline

Pour un fluide, la pression $P$ est fonction de la masse volumique $\rho$ tel que : $P = f(\rho)$ \newline

En se placant dans un milieu homogène constitué uniquement d'air, que l'on approxime comme un gaz parfait, on a, à l'équilibre, on a : $P_{0} = f(\rho_{0})$ \newline

La variation de pression $P_{e}$ due à la source sonore s'exprime de la manière suivante : $P_{e} =  f(\rho_{e})$ 
\footnote{Evidement $P_{e}$ est très petite devant $P_{0}$ (pour le développement de Taylor) et $P = P_{0} + P_{e} = P_{0} + k\rho_{e}$, $\rho = \rho_{0} + \rho_{e}$} \newline

On a finalement : 
\begin{align}
    P &= P_{0} +  P_{e} \nonumber \\
        &= f(\rho_{0} + \rho_{e}) \nonumber \\
    P   &\approx f(\rho_{0}) + \rho_{e}f'(\rho_{0}) \nonumber
\end{align}
On considère maintenant \eqref{eq 1} \newline

On se place au repos ($t=0$) \newline

\begin{itemize}
    \item La position $x$ sur une ligne de courant du fluide s'exprime sous la forme : $\psi (x,t)$. \newline
    \item La position voisine située en $x + dx$ s'exprime sous la forme : $\psi(x + dx,t)$. \newline
    \item La quantité de fluide par unité de surface est définie de la sorte : $\rho_{0}dx$. Avec $dx$ infinitésimal.
\end{itemize}
On obtient : 
\begin{equation}
    \psi(x + dx,t) - \psi(x,t) = \frac{\partial \psi}{\partial x}dx \Longleftrightarrow \rho_{0}dx = \rho(\frac{\partial \psi}{\partial x}dx + dx) \nonumber
\end{equation}
Comme $\rho_{e}$ négligeable devant $\rho_{0}$, on obtient : 
\begin{equation}
    \rho_{e} = - \rho_{0}\frac{\partial \psi}{\partial x} \nonumber
\end{equation}
\newpage
On considère enfin \eqref{eq 3} \newline

On prend une portion du fluide de longueur $dx$. Sa masse est $\rho_{0}dx$ et son accélération $\frac{\partial^{2} \psi}{\partial t^{2}}$. \newline

De plus, on a : 
\begin{equation}
    P_{e}(x,t) - P_{e}(x + dx,t) = \frac{\partial P}{\partial x}dx = -\frac{\partial P_{e}}{\partial x}dx \nonumber
\end{equation}
Soit : 
\begin{eqnarray}
    \rho_{0}\frac{\partial^{2} \psi}{\partial t^{2}} = - \frac{\partial P_{e}}{\partial x}\nonumber
\end{eqnarray}