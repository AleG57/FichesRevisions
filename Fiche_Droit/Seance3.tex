\chapter{Les grandes tendances du droit}
\section{Les contentieux climatiques}
\subsection{L’affaire Grande-Synthe}
\begin{itemize}
    \item CE 19 nov. 2020, n° 427301, Grande-Synthe (Cne)
    \item CE 1er juill. 2021, n° 427301, Grande-Synthe (Cne)
    \item CE, ass., 10 mai 2023, n° 467982,  Grande Synthe
\end{itemize}
\subsection{Le conseil d'état}
Présidé par le premier ministre et à son vice-président
Le conseil d'état a deux grandes missions : \footnote{Les ordres administratifs et judiciaires sont séparés en France}
\begin{itemize}
    \item \textbf{Conseiller le gouvernement} : préparation des lois, ordonnances, certains projets et décrets. \textbf{La cour de cassation est la plus haute entité judiciaire}
    \item \textbf{Juge administratif suprême} : juge les litiges entre l'administration et les administrés, c'est \textbf{la plus haute entité administrative}
\end{itemize}
\newpage
\subsection{Les sections consultatives}
Examination des projets de loi, plusieurs sections :
\begin{itemize}
    \item Section de la \textbf{finance}
    \item Section de l'\textbf{intérieur}
    \item Section \textbf{sociale}
    \item Section des \textbf{travaux publics}
    \item Section de l'\textbf{administration}
\end{itemize}
Les décisions urgentes sont prises par la \textbf{commisison permanente}
\subsection{La section du contentieux}