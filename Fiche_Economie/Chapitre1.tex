\chapter{Introduction : qu'est-ce que la science économique ?}
\section{Définition}
l'analyse économique vient du postulat suivant : \textbf{les besoins de l'homme sont illimité et les ressources limités}. il faut alors faire des concessions lors de la question de l'allocation des ressources. Ce choix implique de renoncer à quelque chose. C'est ce que l'on appelle le \textbf{coût d'opportunité}. \newline

L'économie répond à trois question :
\begin{itemize}
    \item Quoi produire ?
    \item Comment produire ?
    \item Pour qui produire ?
\end{itemize}
\newpage
\section{Economie positive vs normative}
\begin{itemize}
    \item Economie \textbf{positive} : explication objective ou scientifique de l'économie
    \item Economie \textbf{normative} : recommandations fondées sur des jugements personnels (morale, conseils)
\end{itemize}

\section{Macro et micro économie}
\begin{itemize}
    \item \textbf{Macro-économie} : étude de l'économie de manière générale, globale : PIB, taux de chômage ou inflation.
    \item \textbf{Micro-économie} : économie "locale" : ménages, entreprises, Etat, marchés.
\end{itemize}
\section{Analyse conjonturelle et structurelle}
\begin{itemize}
    \item \textbf{Analyse conjoncturelle} : court terme. (ex : comment renforcer la croissance au dernier trimestre 2024 ?)
    \item \textbf{Analyse structurelle} : long terme (ex : économie et transition écologique)
\end{itemize}
\section{l'économie est-elle une science ?}
\begin{itemize}
    \item Econométrie : économie expérimentale
\end{itemize}
Notion de modèle : ensemble d'hypothèses et de lois qui donnent une représentation théorique du fonctionnement de l'économie. \newpage
\section{histoire de l'économie}
\subsection{1776 - 1870 : \textcolor{BrickRed}{Les classiques}}
\begin{itemize}
    \item Adam Smith
    \item David Ricardo
\end{itemize}
l'économie devient autonome de la philosophie. \textbf{La valeur d'un bien est quantifié par la quantité de travail incorporé dans celui-ci : \newline valeur d'un bien = coût de production}. La division du travail est une source d'augmentation de la productivité. les intérêts particuliers servent l'intérêt collectif.
\subsection{1870 - 1920 : \textcolor{BrickRed}{Les néoclassiques (micro)}}
\begin{itemize}
    \item W. Pareto
    \item L. Walras
\end{itemize}
\textbf{La valeur d'un bien provient de l'utilité que l'on tire de sa consommation. Raisonnement à la marge. Valeur d'un bien = utilité du bien}. Introduction des mathématiques dans l'économie.
\subsection{1930 : \textcolor{BrickRed}{Keynes (macro)}}
Critique de la vision néoclassique : \textbf{préconisation des relances budjétaires et monétaires.} En effet, Kaynes considère que laisser le cours de l'économie sans apporter d'aide extérieure ne conduit pas nécessairement au modèle économique optimal. L'état a pour mission de réguler l'économie. \newpage
\subsection{1940 - 1970 : Synthèse néoclassique}
\textcolor{BrickRed}{Synthèse néoclassique = micro néoclassqiue + macro keynésienne}
\begin{itemize}
    \item J.R. Hicks
    \item P. Samuelson
\end{itemize}
Les 30 glorieuses valident la vision keynésienne.
Cependant, après les années 70, à cause du premier choc pétrolier, l'inflation et le chômage remettent en cause les politiques keynésiennes. \newline
\textbf{$Relances \implies \nearrow consommation +  offre = cste \implies inflation \implies \nearrow prix$}
\subsection{1970 : Critique de la vision keynésienne \textcolor{BrickRed}{Monétarisme}}
Selon \textcolor{BrickRed}{les monétaristes}, les effets de relance n'ont pour conséquence que l'inflation. En effet, les consommateurs auraient une anticipation rationnelle : les consommateurs épargnent plus qu'ils ne consomment.
\subsection{1990 : \textcolor{BrickRed}{Nouveaux keynésiens et classiques}}
\begin{itemize}
    \item \textcolor{BrickRed}{les nouveaux keynésiens} : l'intervention de l'état dans les marchés sont nécessaires et efficaces.
    \item \textcolor{BrickRed}{Les nouveaux classiques} : Les individus sont rationnels et les marchés reviennent à l'équilibre : les actions de l'état sont inefficaces et nuisibles.
\end{itemize}
