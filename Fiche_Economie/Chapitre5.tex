\chapter{Economie internationnale et globalisation}
L'économie internationnale se scinde en deux parties :
\begin{itemize}
    \item Le commerce internationnal : l'échange de biens et de services
    \item La finace internationnale : les échanges de capitaux
\end{itemize}
\section{Commerce internationnal}
\subsection{Faits empiriques}
\begin{itemize}
    \item Certains pays ont construits leurs économie grâce au commerce internationnal, comme les pays sud asiatiques par exemple
    \item D'autres pays souffrent du commerce internationnal car ils ont pris du retard, comme par exemple l'Afrique subsaharienne
    \item Dans les pays en interne : certaines régions souffrent plus de l'effet de la mondialisation que d'autres
\end{itemize}
\newpage
\subsubsection{Les accords commerciaux régionnaux}
Question des douanes : une union douanière est-elle meilleure qu'une situation avec des droits de douane semblables pour tous les pays ? \newline
$\implies$ des \textbf{trafics se créent} car des courants commerciaux se forment grâce à l'union et d'autres trafics sont \textbf{détournés} : les courants commerciaux déjà existants se voient détournés par les nouveaux courant commerciaux issus de l'union douanière. \newline

Les droits de douane (coûts) baissent au cours du temps, ce qui favorise les échanges internationnaux

\subsection{L'organisation mondiale du commerce}
\textbf{GATT} : General Agreement on Tarifs and Trade, c'est le code de bonne conduite qui repose sur deux principes
\begin{itemize}
    \item Le \textbf{libéralisme}
    \item Le \textbf{multilatéralisme}
\end{itemize}
L'objectif est d'atteindre un libre échange sans contraintes douanières.
\newpage
\subsubsection{3 Obligations : }
\begin{itemize}
    \item \textbf{Principe de non discrimination} : tout avantage tarifaire accordé à un membre doit être étendu à l’ensemble des membres
    \item \textbf{Principe de réciprocité} : un pays membre ne peut bénéficier des concessions de ses partenaires sans en accorder
    \item \textbf{Principe de transparence} : les barrières non tarifaires doivent être converties en droits de douane afin que leur impact réel devienne transparent
\end{itemize}
\subsubsection{2 Interdictions : }
\begin{itemize}
    \item Le \textbf{dumping}
    \item Les \textbf{subventions}
\end{itemize}
\subsubsection{3 Execeptions : }
\begin{itemize}
    \item Autorisation des accords de libre-échange et des unions douanières
    \item Pays En Développement (PED)
    \item Agriculture et textile
\end{itemize}
\subsubsection{Les vagues de la mondialisation :}
\begin{itemize}
    \item Avant la 1ère guerre mondiale
    \item Après la seconde guerre mondiale \newline
\end{itemize}
\subsubsection{Autres faits :}
Le GATT devient l'OMC depuis 1994 \newline
Les échanges agricoles sont en baisse \newline
Les échanges en énergie sont constants \newline
Les échanges en produit manufacturés en augmentation
\subsection{Théories explicatives}
D'après Adam Smith, "le \textbf{libre-échangisme} est une doctrine économique qui vise à limiter les obstacles à la circulation des biens et des services, et des capitaux entre les économies nationales" \newline
\subsubsection{Les avantages du commerce internationnal}
\large{
\textbf{Théorie des avantages absolus de Smith} : \newline

Un avantage absolu est un bien qu'une nation vend moins cher que les autres (comparaison : monopole d'une firme sur un marché). 
Chaque pays a donc intérêt à se spécialiser dans un domaine précis et de faire baisser ses coûts de production afin de rester concurrentiel sur un marché et profiter des effets du commerce internationnal \newline

\textbf{Théorie des avantages comparatifs de Ricardo} :\newline


\textbf{Tout pays se spécialise dans un domaine qui lui confère le plus grand avantage face aux autres nations}
Un avantage comparatif est avantage lorsque le désavantage d'un bien est moindre comparé à l'avantage d"un autre bien sur les autres nationales. 
La spécialisation internationnale est en réalité bénéfique pour tous car elle permet un acroissement de richesses accrue, même pour un pays désavantagé \footnote{Les Etats-Unis ont-il un intérêt à commercer avec la Chine ? Cf cours chapitre 5 } \newline
\textbf{L'intérêt dans le commerce internationnal réside dans le fait d'avoir un avantage comparativement aux autres nations, plutôt que de raisonner par rapports aux coûts absolus}
}
\newpage
\subsubsection{Théorie des dotations factorielles}
\Large{
\textbf{Dotations en facteurs de production} : «  les nations se spécialisent dans la fabrication qui incorpore le facteur de production qu’elle possède en abondance »
}
\subsubsection{Théorie HOS}

\textbf{La théorie HOS : le libre échange conduit à l'égalisation des coûts de prodution et donc des salaires} \newline

\textbf{Heckscher} : les coûts de production dans chaque pays diffèrent car les facteurs de production y sont différents
\textbf{Ohlin} : la spécialisation d'un pays est basée sur la combinaison de facteurs qui, combinés, donnent un avantage maximum ou bien un désavantage minimum
\textbf{Samuelson} : Dans un monde à deux pays, deux biens et deux facteurs, chaque pays se spécialise dans la production du bien qui requiert relativement \textbf{le plus du facteur le moins cher en autarcie}

\subsubsection{Échanges de similitude}

Echanges internationnaux de même nature : opposition à la théorie de Ricardo et à la loi des dotations de facteurs de production \newline

Pourquoi une telle opposition ?
\newpage
\begin{itemize}
    \item Théorie de Linder : \textbf{la demande représentative}
    \item Théorie de Lassudrie-Duchêne \textbf{la demande de variété}
\end{itemize}
\subsubsection{Théorie de la demande représentative de Linder}
\begin{itemize}
    \item Un produit est conçu pour satisfaire la clientèle internationnale
    \item Pour un produit exporté, les clients potentiels sont de nations ayant un niveau de vie semblable au pays exportateur
    \item L'avantage comparatif est proportionnel à la taille du marché
\end{itemize}
\subsubsection{Théorie de la demande de variété de Lassudrie-Duchêne}
\begin{itemize}
    \item Le commerce intra-branche s'explique par la demande de différence des consommateurs
    \item Les produits ne sont pas rigoureusement identiques ni homogènes (théorie de la concurrence parfaite)
    \item Les consommateurs ont un goût pour la demande de variété
\end{itemize}

\subsubsection{Le protectionnisme}
\textbf{Doctrine économique qui vise à limiter aux étrangers l'accès au marché nationnal} par des droits de douane, quotas ou des contingentementsDoctrine économique qui vise à limiter aux étrangers l'accès au marché nationnal \newline
\footnote{Le protectionnisme permet aux jeunes industries de se démarquer et de survivre face aux firmes implantées dans le marché internationnal grâce aux douanes \newline i.e. l'état peut intervenir dans les phénomènes économiques pour sauvegarder les entreprises nationnales cf. travaux de \textbf{Krugman}, en désaccord avec la théorie \textcolor{BrickRed}{monétariste}, en accord avec la vision \textcolor{BrickRed}{Keynésienne} \newline Plus, permet de favoriser la R\&D et l'innovation}
\subsubsection{Commerce intra-firme}
Commerce entre des FTN \footnote{Firmes Transnationnales} qui mettent en exergue de nouvelles théories du commerce internationnal comme la \textcolor{BrickRed}{concurrence imparfaite}\footnote{La notion "classique" de concurrence y est alors biaisée par ces méthodes de production et imposent des prix de \textbf{pleine concurrence}} et permettent la fragmentation des processus de production
\section{Finance internationnale : les échanges de capitaux}
\subsection{La balance des paiements}