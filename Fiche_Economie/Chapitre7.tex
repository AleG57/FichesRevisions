\chapter{Comptabilité nationnale}
\section{Les identités comptables en économie fermée}
Economie fermée : economie dite nationnale, interne a à un pays, sans interférences avec les autes nations, par définition isolé du commerce internationnal
\begin{center}
    \Large{\fbox{$
    Y = C + I + G
    $}}
\end{center}
\begin{center}
    \Large{\fbox{$
    Y - T = C + S
    $}}
\end{center}
Avec : \begin{itemize}
    \item $Y$ : PIB
    \item $C$ : Consommation privée
    \item $I$ : Investissement privé
    \item $G$ : Dépenses gouvernementales
    \item $T$ : Taxes
    \item $S$ : Epargne
\end{itemize}
\newpage
\section{Interprétation}
La première équation est une équation comptable $\implies$ \textbf{la relation doit être vraie à chaque période} qui signifie que \textbf{tout ce qui est produit doit être dépensé}. Cette équation sert à calculer le PIB \newline
La deuxième équation signifie que le revenu disponible, c'est à dire le PIB retranché aux taxes, est redistribué à la société, qui peut soit consommer l'argent redistribué soit en garder une partie sous forme d'épargne.
\section{Conséquences}
Il vient de ces deux équations : 
\begin{center}
    \Large{\fbox{$
    I + G = T + S
    $}}
\end{center}
\begin{itemize}
    \item Si $(I + G) - S$ est positif : représente le \textbf{besoin} de financement de l'économie
    \item Si $(I + G) - S$ est négatif : représente le \textbf{déficit public} \footnote{= Plus d'argent redistribué que d'argent produit}
    \item Si $(I + G) = S$ : équilibre de financement de l'économie
\end{itemize}
\begin{itemize}
    \item Si $G + T$ négatif $\implies$ \textbf{déficit public}
    \item Si $G + T$ positif $\implies$ \textbf{exédent public} \newline
\end{itemize}
\newpage
A l'équilibre on a : 
\begin{center}
    \Large{\fbox{$
    I + G = S
    $}}
\end{center}
Soit : 
\begin{center}
    \Large{\fbox{$
    I + G = Y - T -C
    $}} \footnote{Equilibre emploi-ressources}
\end{center}
Si $G = T = 0$ alors $S = I = Y - C$ : la richesse est soit investie soit consommée \newline
\section{Les identités comptables en économie ouverte}
L'économie ouverte considère le commerce internationnal
\begin{center}
    \Large{\fbox{$
    Y + M = C + I + G + X
    $}}
\end{center}
\begin{itemize}
    \item $M$ : importations
    \item $X$ : exporations
\end{itemize}
\newpage
\section{Remarques liées au TD}
\begin{itemize}
    \item les biens anciennement produits (ex : appartement vieux de 20 ans) ne sont pas pris en compte dans le calcul du PIB, la calcul de $Y$ est donc caduque 
    \item Par convention comptable, les frais de scolarité rentrent dans la \textbf{consommation}
\end{itemize}